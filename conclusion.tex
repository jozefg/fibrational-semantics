\section{Conclusion}

In these notes we have really barely scratched the surface of what
fibrations have to offer to categorical semantics. The interested
reader is referred to~\citet{Jacobs:01} for a book length treatment of
the subject. Much the covered semantics are a more detailed
explanations described in chapter 8.

Beyond merely describing type theories, fibrations have a seen a lot
of use recently in describing the semantics of polymorphism. Reynolds
Abstraction Lemma~\cite{Reynolds:84} when properly phrased can be
shown to coincide precisely with the statement that an appropriate
natural transformation is fibered. This theory has been developed
recently by~\citet{Ghani:15} and Patricia Johann's introductory notes
on the subject may be found at
\url{http://www.cs.appstate.edu/~johannp/oplss.html}. I would love to
write another set of notes on this work but that would require me to
understand it first.

By providing an abstracted notion of indexing we have shown that
fibrations provide a natural basis for categorical semantics for a
large number of languages. Indeed they can serve to generalize several
of the existing methods for modeling semantics as well as providing a
natural way of layering a logic or type theory on top of a base type
theory.

I hope you have found this helpful.
