
\section{An Introduction to Fibrations}

Before diving immediately into the categorical formulation of
fibrations I would like to start with the set-theoretic intuition. In
set theory, families of sets are a common occurrence. They are usually
denoted $(X_i)_{i \in I}$ so that for any element $i \in I$ we have a
designated set $X_i$. This presentation is what we shall refer to as
the \emph{index-based} presentation of families of sets. An equivalent
and useful way to view this family is as a function
\[
  \iota : X = \bigcup_{i \in I} X_i \to I
\]
we will refer to $X$ as the \emph{total} set and $I$ as the
\emph{index} set. The idea is that every element $x \in X_i$ instead
becomes $(i, x) \in X$, a pair of the index at which it lives and the
actual element. The map $\iota$, the so called \emph{presentation}
map, will then send $(i, x) \mapsto i$. We can then easily recover our
original $X_i$s from $\iota$.
\[
  X_i = \iota^{-1}(i)
\]
For a user of set theory this presentation may seem slightly stilted
but it does have the major advantage that it's purely based on maps
between sets. This is something we can conceivably generalize to
categories since categories possess this structure. The next question
what properties of presentation maps to we wish to preserve when we
categorify it.

One feature that will turn out to be of great importance is the
\emph{reindexing} of families of sets. For sets, if we have a map
$f : I \to J$ and $(X_i)_{i \in I}$ we can conceivably form a
family of sets which is indexed by $J$, $(Y_j)_{j \in J}$, where $Y_j$
is defined as
\[
  Y_i = X_{f(i)}
\]
This reindexing again only depends on maps between sets so it is quite
natural to categorify. The first way to categorify this is to maps
sets to objects and functions as morphisms. We then define a family as
a morphism $f : B \to A$. We generally will write this as \emph{$B$
  over $A$} or pictorially
\[
  \begin{tikzcd}
    B \ar[d, "f"] \\
    A
  \end{tikzcd}
\]
We cannot naturally recover what $B_a$ might mean yet because we do
not have the same notion of elements that we do in set
theory. Instead, we will see later that this is a general result of
reindexing.

Reindexing reappears here quite naturally in finite limit categories
in the form of pullbacks. Specifically, if we cast a set theoretic
function $f : X \to Y$ as a morphism $f : I \to J$, given a family
$b : A \to I$ we can define the reindexing $f^*(b)$ as the pullback
\[
  \begin{tikzcd}
    f^*(A) \ar[d, swap, "f^*(b)"] \ar[r] & A \ar[d, "b"]\\
    I \ar[r, "f", swap] & J
  \end{tikzcd}
\]

Now that we have generalized the notion of indexing from sets, we can
start to observe some of the categorical structures which arise from
this \emph{indexing-as-presentations} approach. First, families over
$A$ for some given $A$ quite naturally form a category: $\Ccat/A$.
\begin{defn}\label{defn:fibrations:slicecat}
  The \emph{slice category} over $A$, written $\Ccat/A$ is defined as
  the category with
  \begin{description}
  \item[Objects] Objects are families $f : B \to A$
  \item[Morphisms] A morphism between $f : B \to A$ and $g : C \to A$ is
    a morphism of $h \in \hom_{\Ccat}(B, C)$ so that $f = g \circ h$.
    \[
      \begin{tikzcd}
        B \ar[rr, "h"] \ar[dr, swap, "f"] & & B \ar[dl, "g"]\\
        & A &
      \end{tikzcd}
    \]
  \end{description}
\end{defn}
Now that we have categories for families over $A$, it is natural to
ask how the reindexing maps $f^*$ act on these categories. In
particular, do these respect the structure of slice categories and
thus form a functor. As it turns out, this is the case.
\begin{thm}\label{thm:fibrations:pullbackfunctor}
  For any $f : A \to B$ in a finite limit category, reindexing along
  $f$ gives a functor $f^* : \Ccat/B \to \Ccat/A$.
\end{thm}
\begin{proof}
  TODO
\end{proof}

This sort of structure, categories of indexed families with reindexing
functors between them will turn out to be a natural setting for models
of type theory. To give a flavoring for why this is interesting, we
will model each $\ctxJ{\Gamma}$ as an object our category, each
$\typeJ{\Gamma}{A}$ as a family over $\denote{\ctxJ{\Gamma}}$ and
terms as morphisms between these families. The reindexing functor
correspond naturally to substitution. With this we reduce the problem
of axiomatizing our semantics to merely demanding the existence of
certain categorical constructs in each slice category. In other words,
we will use the categorified indexing to represent how types and terms
depend on contexts. By abstracting away this structure now then, we
can easily wipe away half the work of defining categorical semantics.

In order to generalize away from one particular form on indexing, we
isolate the idea of having a having a family of categories varying
over a base or index category. This is a fibration. I will start by
giving the formal definition of a fibration which makes references to
some as of yet undefined concepts. I will then progressively attempt
to fill it in so that the original, formal, definition becomes more
intuitive.
\begin{defn}\label{defn:fibration:fibration}
  A fibration, $p : \Ecat \to \Bcat$, so that for every morphism
  $u \in \hom_{\Bcat}(I, p(Y))$ there is a cartesian lifting
  $\bar{u} : u^*(Y) \to Y$.
\end{defn}
Intuitively a fibration is a way of gluing a collection of categories
indexed by another category so we start with a \emph{display functor}
$p : \Ecat \to \Bcat$. We will think of $\Ecat$ as the collection of
all the objects and morphisms of our indexed categories clumped
together and $p$ tells us the index of some particular
object. Accordingly, we can then say that a morphism,
$f \in \hom_{\Ecat}(A, B)$ is vertical if $p(f) = 1$. It is obvious
that $\Ecat_I$ comprised of objects $X$ so that $p(X) = I$ and
morphisms vertical over $I$ forms a category. It is important that
$\Ecat$ is not necessarily just $\amalg_I \Ecat_I$, the morphisms that
are not vertical will important for relating the structure $\Bcat$ to
$\Ecat_I$.

What we would like next is a way from moving between $\Ecat_I$ and
$\Ecat_J$ using the information we have from $\hom_\Bcat(J, I)$. This
is just a generalization of what we did with slice categories to
fibrations. In order to do this, given a map $f : J \to I$ we wish to
find a way to transport from $\Ecat_I$ to $\Ecat_J$. In order to do
this, we wish to lift the morphism in some way so that given $X$ so
that $p(X) = I$ we have a map $X \to Y$ for some $Y$ so that
$p(Y) = J$ that projects down onto $f$. If we think of this as
substitution over contexts (like our above example) this roughly
corresponds to saying that if we have a way to chance our context to a
different context, we can change types in the first context to types
in the second. Something that seems quite plausible, it ought to be
just substitution on types.

It would be ideal if this morphism was canonical in some sense. It's
quite easy to imagine a scenario where there are infinitely many
different morphisms that we could lift to. If we think of this
operation as corresponding to substitution, then there must be a
canonical choice for us to pick. Now for some notation, we write this
hypothetical \emph{best substitution} as a morphism in $\Ecat$,
$\bar{f} : f^*(X) \to X$ in order to evoke the similarity to the
pullback case we started with. As is usual in category theory, we will
characterize this lifting using a universal property. Namely, we want
this morphism to be terminal. This means that given
\[
  \begin{tikzcd}
    J \ar[r, "f"] \ar[d, "h"] & p(X)\\
    p(Y') \ar[ru, "p(g)", swap]
  \end{tikzcd}
\]
we want a canonical map
\[
  \begin{tikzcd}
    f^*(X) \ar[r, "\bar{f}"] & p(X)\\
    Y' \ar[u, dashed] \ar[ru, "g", swap]
  \end{tikzcd}
\]
in this case, our lifting $\bar{u}$ is said to be \emph{cartesian}.
\begin{defn}\label{defn:fibration:cartesian}
  A map $f : X \to Y$ is said to be cartesian over $u : f(Y) \to f(X)$
  if $p(f) = u$ and if for any map $g : Y' \to X$ so that
  $u \circ v = p(g)$, there is a unique $h : Y' \to Y$ so that
  $p(h) = v$ and $f \circ h = g$.
\end{defn}

With this we have at least acquired the necessary technical
information to understand what a fibration is, however we're still
lacking in some intuition. In particular, the connection between
cartesian maps and reindexing functors remains unclear. What we really
wanted was a family of functors $u^*$ for each $u : J \to I$ mapping
$\Ecat_I$ to $\Ecat_J$. Well as it happens, this functor is definable
using cartesian maps.

\begin{thm}\label{thm:fibrations:reindexing}
  For a fibration $p : \Ecat \to \Bcat$, any map $u : J \to I$ in
  $\Bcat$ induces a functor $u^* : \Ecat_I \to \Ecat_J$.
\end{thm}
\begin{proof}
  todo
\end{proof}

In order to see that this characterization is even equivalent to this
sort of indexing structure, we will next investigate the so called
\emph{Grothendieck construction}. This allows us to turn a
pseudofunctor $\Bcat \to \cat{Cat}$ into a particular fibration. This
lets us know that we haven't lost or gained anything essential by
switching to the fibrational presentation of indexing.

\begin{thm}\label{thm:fibrations:grothendieckconstruction}
  Every pseudofunctor $\Bcat \to \cat{Cat}$ corresponds to a fibration
  over $\Bcat$.
\end{thm}
\begin{proof}
  todo
\end{proof}

Having set out this very abstract definition, I will now relate it to
what we were previously studying: indexed families of sets. In
particular, let us look at the so called \emph{codomain fibration}:
$\cod$. In order to do this, let us start with a few preliminary
definitions.

\begin{defn}\label{defn:fibrations:arrowcat}
  The arrow category over a category, $\Ccat$, is written
  $\Ccat^{\to}$ and is defined with
  \begin{description}
  \item[Objects] For objects arrows $f : A \to B$
  \item[Morphisms] A morphism between $f : A \to B$ and $g : C \to D$
    is a pair of morphisms $(h, h')$ so that $h : A \to C$ and $h' : B
    \to D$ and the following commutes
    \[
      \begin{tikzcd}
        A \ar[d, "f", swap] \ar[r, "h"] & B \ar[d, "g"]\\
        C \ar[r, "h'", swap] & D
      \end{tikzcd}
    \]
  \end{description}
\end{defn}

The reader is referred to~\citet{MacLane:98} for a more detailed
investigation of the arrow category. For our purposes we are largely
interested in the codomain functor $\cod : \Ccat^{\to} \to
\Ccat$. It is defined
\begin{align*}
  \cod(f : A \to B) &= B\\
  \cod(h, h') &= h'
\end{align*}
it is trivial to check this indeed defines a functor. More than this
though, in a category with pullbacks this functor is in fact a
fibration.
\begin{example}\label{example:fibrations:cod}
  The $\cod$ functor defines a fibration.
\end{example}
\begin{proof}
  todo
\end{proof}

Now an important step of understanding a fibration
$p : \Ecat \to \Bcat$ is to study the fibers that it defines,
$\Ecat_I$. In particular, we shall find this illuminating in the case
of the codomain fibration

\begin{lem}\label{lem:fibrations:fibersareslices}
  The fibers of $\cod : \Ccat^{\to} \to \Ccat$ are precisely the slice
  categories of $\Ccat$. Moreover, reindexing in the fibration
  corresponds to the pullback functor of slice categories.
\end{lem}
\begin{proof}
  todo
\end{proof}

Have done this we have come full circle: our very abstract notion of
fibration can be specialized to simple set theoretic indexing by
instantiating a particular fibration on a particular category.

Having shown that our abstraction is a useful one, we can develop the
theory of fibrations abstractly. The reader is referred to
\citet{Jacobs:01} for a more in-depth coverage of fibered category
theory. For our purposes, I only want to flesh out enough theory to
explain the semantics of System F in the next section.

The reindexing functors are our first object of consideration. In
particular, it is worth mentioning a few details that I would like to
promptly handwave away. Firstly, the definition of a fibration
stipulates the existence of liftings but does not include the data of
which particular liftings we chose. If one wants to model this set
theoretically, one needs to demand the axiom of choice or select a
particular collection of liftings. The alternative is to not use a
broken notion of existential quantification but that seems less likely
to see widespread adoption. In any case, a fibration is said to be
\emph{cloven} if it comes equipped with a choice of liftings. Next the
liftings themselves give rise to reindexing functors as we have
seen. However, there is a question as to the coherence of these
functors. For instance, we have seen that
$u^* \circ v^* \cong (v \circ u)^*$ and $1 \cong 1^*$. In a fibration
if our choice of liftings makes these equivalences simply identities,
we shall call the fibration \emph{split}. Now it is the case that
every cloven fibration is equivalent to a split one so I shall for the
rest of this paper assume that we are working with split cloven
fibrations unless stated otherwise.

The next area to explore is the structure that may be found in each
fiber. In particular, we might want to demand the usual categorical
constructs: limits, colimits, exponentials, etc. It turns out to be
straightforward to add these to our model.

\begin{defn}\label{defn:fibrations:fiberwiseX}
  A fibration $p : \Ecat \to \Bcat$ is said to posses \emph{fiberwise
    X} if each fiber $\Ecat_I$ possesses Xs and Xs are preserved by
  reindexing. We will say that this structure is \emph{split} if it is
  preserved on the nose by reindexing.
\end{defn}

The final piece of structure that we shall need for our model is
something to handle the impredicative polymorphism of System F. For
this, we need an object in our index category which classifies types
in the total category. This is called a \emph{generic object}.

\begin{defn}\label{defn:fibrations:genericobject}
  A generic object in a fibration $p : \Ecat \to \Bcat$ if an object
  $T \in \Ecat$ so that for any $X \in \Ecat$, there is a unique map
  $u : p(X) \to p(T)$ so that there exists an $f : X \to T$ so that
  $f$ is cartesian~\ref{defn:fibrations:cartesian} over $u$.
\end{defn}

The intuition here is that given any type, represented as an object in
the category $\Ecat$, we can find a unique element $p(X) \to p(T)$ so
that applying this substitution to $T$ gives us back the original type
we started with. This tells us that cartesian liftings of
$\hom(p(X), p(T))$ are in bijection with the objects of
$\Ecat_X$. This will be used to model a ``type of small types'' for
impredicative polymorphism.
