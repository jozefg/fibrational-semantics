\section{Introduction}

Let us first start with a brief discussion motivating categorical
semantics in general and fibrations in particular. Denotational
semantics is a rich and well studied area of type theory dating back
to work by Scott and Strachey. The core idea is to reduce the studying
of programs to the study of better understood mathematical
objects. Originally these tended to be things like \emph{sets} or
\emph{continuous lattices}. As time has evolved though we have shifted
to objects such as \emph{games} or other exotic objects. Lately
however the field of categorical semantics has exploded. Categorical
semantics in some sense acts as glue between syntax and traditional
denotational semantics.

With categorical semantics one doesn't use concrete. Instead,
categorical semantics aim to axiomatize the structure necessary for
any concrete model to be to possess. The classic example is that of a
cartesian closed category (CCC). This possesses enough structure to model
precisely the simply typed lambda calculus. Now the power of
categorical semantics becomes apparent, by proving that we can model
the simply typed lambda calculus in any CCC, we have suddenly reduced
the problem of finding models to the problem of finding CCCs. This is
a comparatively easy and well-studied problem. In fact, this
immediately gives us dozens of models of the simply typed lambda
calculus for free
\begin{enumerate}
\item $\cat{Set}$
\item $\cat{Dom}$
\item $\cat{Eq}$
\item $\cat{PEq}$
\item ...
\end{enumerate}
This illustrates the role of categorical semantics. They mediate
between the purely syntactic demands of a type theory and the
mathematically natural constraints that we need to impose on a
category to obtain a model. By finding a convenient set of constraints
for a model, models become cheap and easy to find.

This is the role of categorical semantics but what about fibrational
semantics? Fibrations are the categorical version of
indexing. Indexing is a prevalent phenomenon in type theory; terms
depend on types, types may depend on other types, types may even
depend on terms! It turns out that be starting with a notion of
indexing and imposing constraints on the indexed and the indexing
categories we can quickly and flexibly model type theory. Again we
encounter the advantages of generality become apparent when working
with fibrational models. By specifying the properties we require of a
fibration to provide in order to support a model of a type theory, we
generalize over even the variety of fibrations that support this. For
instance, if a split fibration yields a model of predicate logic then
we are not merely limited to one source of fibrations. Throughout
these notes the generality that this provides will make it apparent
how fibrational semantics generalize the ideas of models into locally
cartesian closed categories, toposes, and regular categories.
