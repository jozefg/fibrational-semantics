\section{Fibrational Semantics Compared to Other Approaches}

In the wide world of categorical semantics its worthwhile to pause and
ask what distinguishes fibrational semantics as a useful tool.
This section examines a few other common category theoretic tools used
modeling type theories and discusses their relationship to fibrations.

Perhaps the mostly widely used structure for modeling simple type
theories is a closed cartesian category. In fact in this scenario I
would expect fibrations to be a limited utility. There is not a
complicated indexing or dependency structure in this logic. Therefore,
having high powered machinery for handling indexing is simply a
waste. In fact, for any propositional logic or type theory without
type dependency of some sort fibrations are not nearly as useful. In
this sense, fibrations are incomparable to CCCs and similar tools.

A more interesting example is that of locally cartesian closed
categories. These categories provide models of Martin-L\"of Type
Theory~\cite{MartinLof:84}.
\begin{defn}\label{defn:othersemantics:lccc}
  A locally cartesian category $\mathcal{E}$ is a category where each
  slice category is cartesian closed. Equivalently, the reindexing
  functor $f^*$ has both a left and right adjoint.
\end{defn}
Under the LCCC methodology, contexts are interpreted as objects in the
category and types are interpreted as slices over their contexts.
\[
  \begin{tikzcd}
    \denote{\Gamma, A} \ar[d, "\denote{A}"]\\
    \denote{\Gamma}
  \end{tikzcd}
\]
The cartesian closer of the slice categories means that we can form
function types using exponentials and quantification is interpreted by
the various adjoints to $f^*$ under the quantifiers are adjoints
slogan. However, LCCCs are a strict special case of fibrations. They
are simply categories for which the $\cod$ fibration is a fiberwise
CCC~\ref{defn:fibrations:fiberwiseX} or equivalently that it has
products and coproducts~\ref{defn:fibrations:products}. By relaxing
the particular fibration we require this definition generalizes
LCCCs. A direct fibrational generalization of LCCCs to capture models
of Martin-L\"of type theory are comprehension
categories~\cite{Jacobs:93}. These are functors $P : \Ecat \to
\Bcat^{\to}$ so that $\cod \circ P$ forms a fibration. By postulating
various simple properties about this fibration this forms a model of
type theory.

Another common basis for categorical logic is topos theory.
\begin{defn}\label{defn:othersemantics:topos}
  An elementary topos $\Etop{}$ is a category that
  \begin{itemize}
  \item has all finite limits
  \item is cartesian closed
  \item has a subobject classifier, $\Omega$
  \end{itemize}
\end{defn}
The study of elementary topoi is a rich subject. The reader is referred
to~\citet{MacLane:94} or~\citet{Johnstone:14} for a more in depth
account. The high level of idea of topos theoretic models, when not
simply using them as more structured LCCCs, is to use the subobject
classifier as an internal universe of sorts. $\Omega$ forms an
internal Heyting algebra and thus a model of intuitionistic logic. It
also comes equipped with quantifiers via adjoints (they are derivable)
to pullback functors as with LCCCs. By interpreting predicates on an
object as subobjects of it
\[
  \begin{tikzcd}
    R \ar[d, rightarrowtail, "m"]\\
    A \times A
  \end{tikzcd}
\]
we can interpret many sorted higher order intuitionistic logic into a
topos. The higher order aspect comes from exponentiation on
$\Omega^A$, which is equivalent to predicates on $A$ since
$\hom(A, \Omega) \cong \sub(A)$. Of particular interest are
Grothendieck topoi. These are topoi formed as sheaves on a particular
category equipped with a Grothendieck topology. These provide a
categorical analog generalizing forcing for instance. However, topoi
too may be classified by fibrations. An topos is elementary topos is
simply a category whose subobject fibration $\sub$ has fibered
products and exponentials as well as a generic object. This
classification describes the topos theoretic approach to interpreting
logic. It explains how predicates in context are understood as
subobject and isolates $\Omega$ as the object level reflection of this
interpretation. Again by generalizing this strategy of indexing we can
find alternative models of higher order intuitionistic logic anywhere
we are able to construct a higher order fibration.
