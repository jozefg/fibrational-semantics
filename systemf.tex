\section{Semantics for System F}

Let us start with a simple definition of System F. To be more explicit
than usual, let us start by noting the three core judgments of the
system.
\[
  \ctxJ{\Delta}{\Gamma} \qquad \typeJ{\Delta}{\tau} \qquad \hastypeJ{\Delta}{\Gamma}{e}{\tau}
\]
The judgment for $\ctxJ{\Delta}{\Gamma}$ is a simple extension of the
typehood judgment.
\begin{mathpar}
  \inferrule{ }{\ctxJ{\Delta}{\cdot}} \and
  \inferrule{
    \ctxJ{\Delta}{\Gamma}\\
    \typeJ{\Delta}{\tau}
  }{\ctxJ{\Delta}{\Gamma, x : \tau}}\and
  \inferrule{
    \typeJ{\Delta, \alpha}{\tau}
  }{\typeJ{\Delta}{\all{\alpha}{\tau}}}\and
  \inferrule{
    \typeJ{\Delta}{\tau_1}\\
    \typeJ{\Delta}{\tau_2}
  }{\typeJ{\Delta}{\tau_1 \to \tau_2}}\and
  \inferrule{
    \alpha \in \Delta
  }{\typeJ{\Delta}{\alpha}}\and
\end{mathpar}

Next we have the usual typing rules for terms. One will observe that I
deviate from some presentations of System F by using two contexts to
represent types \emph{and} terms rather than blending them into one
context. This is a nod towards the semantics where such things will be
fundamentally separated anyways, with one living in the base category
and one living in the fibers above it.

\begin{mathpar}
  \inferrule{
    x : \tau \in \Gamma
  }{\hastypeJ{\Delta}{\Gamma}{x}{\tau}}\and
  \inferrule{
    \hastypeJ{\Delta}{\Gamma, x : \tau_1}{e}{\tau_2}
  }{\hastypeJ{\Delta}{\Gamma}{\lam{x}{\tau}{e}}{\tau_1 \to \tau_2}}\and
  \inferrule{
    \hastypeJ{\Delta}{\Gamma}{e_1}{\tau_1 \to \tau_2}\\
    \hastypeJ{\Delta}{\Gamma}{e_2}{\tau_1}
  }{\hastypeJ{\Delta}{\Gamma}{\ap{e_1}{e_2}}{\tau_2}}\and
  \inferrule{
    \hastypeJ{\Delta, \alpha}{\Gamma}{e}{\tau}
  }{\hastypeJ{\Delta}{\Gamma}{\Lam{x}{e}}{\all{\alpha}{\tau}}}\and
  \inferrule{
    \hastypeJ{\Delta}{\Gamma}{e}{\all{\alpha}{\tau_1}}\\
    \typeJ{\Delta}{\tau_2}
  }{\hastypeJ{\Delta}{\Gamma}{\Ap{e}{\tau_2}}{[\tau_2/\alpha]\tau_1}}\and
\end{mathpar}

Having fleshed out the objects of our language we now turn to the
semantics of our language. They will proceed in three parts. First we
isolate the specific variety of fibration that we will be
studying. Second we will denote the terms and contexts into objects in
base category. Finally, we will denote terms into morphisms in the
fibers above contexts. Put briefly
\begin{description}
\item[Type contexts] objects in the base category.
\item[Types] morphisms in the base category into the generic object
  from the type context. Equivalently, objects in the fiber above the
  type context.
\item[Term Contexts] objects in the fiber above a type context.
\item[Terms] morphisms between term contexts. Equivalently, vertical
  morphisms in the fiber above the type context.
\end{description}

With this plan in mind, we are in a position to define what structure
we will need for our fibration.
\begin{defn}\label{defn:systemf:polymorphicfibration}
  A polymorphic fibration, $p : \Ecat \to \Bcat$ is a fibration with
  \begin{enumerate}
  \item Fiberwise finite products and exponentials.
  \item Finite products in the base category.
  \item A generic object $T$.
  \item Right adjoints to the reindexing functors for projections.
  \end{enumerate}
\end{defn}
For now we will set aside the question of what fibrations exist which
satisfy these constraints and focus instead show that this is
sufficient to model System F. First, we present the denotation of
type context.
\begin{align*}
  \denote{\cdot} &= 1\\
  \denote{\Delta, \alpha} &= \denote{\Delta} \times p(T)
\end{align*}
This is just the straightforward mapping based on the insight that
maps into $p(T)$ will classify types via transposition. That is, we
know that there is a bijection
$\phi_I : \hom(I, p(T)) \cong \obj(\Ecat_I)$.  Next comes the denotation
of types.
\begin{align*}
  \denote{\typeJ{\Delta}{\alpha}} &= \pi_{\alpha}\\
  \denote{\typeJ{\Delta}{\tau_1 \to \tau_2}} &=
  \phi^{-1}\left(\phi(\denote{\typeJ{\Delta}{\tau_2}})^{\phi(\denote{\typeJ{\Delta}{\tau_1}})}\right)\\
  \denote{\typeJ{\Delta}{\all{\alpha}{\tau}}} &=
  \phi^{-1}\left(\Pi_{\pi}\phi(\denote{\typeJ{\Delta, \alpha}{\tau_1}})\right)
\end{align*}
Here $\pi_{\alpha}$ is the projection extraction $\alpha \in \Delta$
from the product $\denote{\Delta}$. Notice the use of $\phi$ to make
use of the structure that we have at the level of fibers in the base
category. This ability to tie together the base and total category is
a common usage of generic objects. An important lemma that we shall
make use of is that implicit weakening of typing derivation
corresponds to categorical weakening (reindexing by $\pi^*$).

\begin{lem}\label{lem:systemf:weakening}
  If $\typeJ{\Delta}{\tau}$ then
  $\denote{\typeJ{\Delta, \alpha}{\tau}}$ factors through
  $\denote{\typeJ{\Delta}{\tau}}$ by
  $\pi_1 : \denote{\Delta} \times p(T) \to \denote{\Delta}$.
\end{lem}
\begin{proof}
  We proceed by induction on $\typeJ{\Delta}{\tau}$
  \begin{itemize}
  \item
    $\inferrule{
      \beta \in \Delta
    }{\typeJ{\Delta}{\beta}}$

    In this case we know that $\beta \neq \alpha$ since
    $\alpha \not\in \Delta$. Next we note that
    $\denote{\typeJ{\Delta, \alpha}{\beta}} = \pi_\beta$. Now we know
    that $\pi_\beta = \pi_\beta \circ \pi$ so we're done.
  \end{itemize}
\end{proof}

Now by the definition~\ref{defn:fibrations:genericobject} we know that
this means precisely that
\[
  \pi_1^*(\phi(\denote{\typeJ{\Delta}{\tau}})) = \phi(\denote{\typeJ{\Delta, \alpha}{\tau}})
\]
so weakening does precisely correspond to categorical weakening. This
is important because we want to use the adjunction
$\pi^* \dashv \Pi_\pi$. However, in order to do this we need a
morphism with domain $\pi^*(A)$ for some $A$. The above lemma tells us
that when it's $\phi(\denote{typeJ{\Delta, \alpha}{\tau}})$ where we
happen to know that $\typeJ{\Delta}{\tau}$ holds then this is the case.

Now under $\phi$, we know that the denotation
$\denote{\typeJ{\Delta}{A}}$ corresponds to an object in the fiber
above $\denote{\Delta}$. Using this, we shall denote terms
$\hastypeJ{\Delta}{\Gamma}{e}{\tau}$ into morphisms in
\[
  \hom_{\Ecat_{\denote{\Delta}}}(\denote{\ctxJ{\Delta}{\Gamma}}, \phi(\denote{\typeJ{\Delta}{A}}))
\]
This is defined, as is standard, by induction on the typing
derivation. In order to do this, let us shall write $\psi_A$ for the
natural bijection $\hom(\pi^*(A), B) \cong \hom(A, \Pi B)$
\begin{align*}
  \denote{\hastypeJ{\Delta}{\Gamma}{x}{\tau}} &= \pi_x\\
  \denote{\hastypeJ{\Delta}{\Gamma}{\lam{x}{\tau_1}{e}}{\tau_1 \to \tau_2}} &=
  \Lambda(\denote{\hastypeJ{\Delta}{\Gamma, x : \tau_1}{e}{\tau_2}})\\
  \denote{\hastypeJ{\Delta}{\Gamma}{\ap{e_1}{e_2}}{\tau_2}} &=
  \ev \circ
  \langle \denote{\hastypeJ{\Delta}{\Gamma}{e_1}{\tau_1 \to \tau_2}}, \denote{\hastypeJ{\Delta}{\Gamma}{e_2}{\tau_1}} \rangle\\
  \denote{\hastypeJ{\Delta}{\Gamma}{\Lam{\alpha}{e}}{\all{\alpha}{\tau}}} &=
  \psi(\denote{\hastypeJ{\Delta, \alpha}{\Gamma}{e}{\tau}})\\
  \denote{\hastypeJ{\Delta}{\Gamma}{\Ap{e}{\tau_1}}{[\tau_1/\alpha]\tau_2}} &=
  \langle 1, \denote{\typeJ{\Delta}{\tau_1}}\rangle^*(\denote{\hastypeJ{\Delta, \alpha}{\Gamma}{e}{\all{\alpha}{\tau_2}}})
\end{align*}
Again we are using $\pi_x$ to refer to the projection removing
$\denote{\typeJ{\Delta}{\tau}}$ from
$\denote{\ctxJ{\Delta}{\Gamma}}$. Notice that we are making use of
lemma~\ref{lem:systemf:weakening} in order to apply the transposition
$\psi$ above. We now want to show soundness, that is, that
$\denote{-}$ respects $\beta\eta$-equivalence. Let us first explicitly
spell out the rules that we expect to hold.
\begin{mathpar}
  \inferrule{
    \hastypeJ{\Delta}{\Gamma}{e}{\tau}
  }{
    \hastypeJ{\Delta}{\Gamma}{e \equiv e}{\tau}
  }\and
  \inferrule{
    \hastypeJ{\Delta}{\Gamma}{e_1 \equiv e_2}{\tau}
  }{
    \hastypeJ{\Delta}{\Gamma}{e_2 \equiv e_1}{\tau}
  }\and
  \inferrule{
    \hastypeJ{\Delta}{\Gamma}{e_1 \equiv e_2}{\tau}\\
    \hastypeJ{\Delta}{\Gamma}{e_2 \equiv e_3}{\tau}
  }{
    \hastypeJ{\Delta}{\Gamma}{e_1 \equiv e_3}{\tau}
  }\and
  \inferrule{
    \hastypeJ{\Delta}{\Gamma}{e_1 \equiv e_2}{\tau_1}
    \hastypeJ{\Delta}{\Gamma, x : \tau_1}{C[x]}{\tau_2}
  }{
    \hastypeJ{\Delta}{\Gamma}{C[e_1] \equiv C[e_2]}{\tau_2}
  }\and
  \inferrule{ }{
    \hastypeJ{\Delta}{\Gamma}{\lam{x}{\tau_1}{\ap{e}{x}} \equiv e}{\tau_1 \to \tau_2}
  }\and
  \inferrule{ }{
    \hastypeJ{\Delta}{\Gamma}{\ap{(\lam{x}{\tau_1}{e_1})}{e_2} \equiv [e_2/x]e_1}{\tau}
  }\and
  \inferrule{ }{
    \hastypeJ{\Delta}{\Gamma}{\Lam{\alpha}{\Ap{e}{\alpha}} \equiv e}{\all{\alpha}{\tau}}
  }\and
  \inferrule{ }{
    \hastypeJ{\Delta}{\Gamma}{\Ap{(\Lam{\alpha}{e})}{\tau} \equiv [\tau/\alpha]e}{\tau}
  }
\end{mathpar}
Now the first three equivalences are trivial to validate because the
ambient equality in a category is an equivalence relation. Therefore,
the other four rules are the interesting ones to prove.
\begin{thm}\label{thm:systemf:soundness}
  If $\hastypeJ{\Delta}{\Gamma}{e_1 \equiv e_2}{\tau}$ then
  $\denote{\hastypeJ{\Delta}{\Gamma}{e_1}{\tau}} = \denote{\hastypeJ{\Delta}{\Gamma}{e_2}{\tau}}$
  as morphisms in the fiber above $\denote{\Delta}$.
\end{thm}
\begin{proof}
  todo
\end{proof}

Now that we know that these fibrations will give us a sound model of
System F, we turn to the question of what examples exist in the
wild. For the sake of concision, I will only present one example in
depth. That the fibration of families of PERs gives a model. Firstly,
we start by defining the relevant constructions.
\begin{defn}\label{defn:systemf:assembly}
  An assembly, $X$, \footnote{We will specialize this definition to
    $\mathcal{K}_1$ for simplicity but an arbitrary PCA suffices} is a
  pair of a set $\card{X}$, and a map $E : X \to \pow{\mathbb{N}}$ so
  that for every $x \in X$, $E(x) \neq \emptyset$.
\end{defn}

We shall say that $n \in E(x)$ \emph{realizes} $x$ and write this more
succinctly as $n \Vdash_X e$. It will be helpful to remember that we
can encode Turing Machines\footnote{Or lambda terms if one wishes to
  be a bit more civilized.} as natural numbers. This surjective
encoding gives rise to a partial application operation, $\cdot$ where
$n \cdot m = k$ if and only if the $n$th Turing machine when fed the
input $m$ returns $k$. We can say that a realizer \emph{tracks} a map
$n \Vdash f$ if
\[
  m \Vdash x \implies n \cdot m \Vdash f(x)
\]
This definition lets us form a categorical structure from assemblies.
\begin{defn}\label{defn:systemf:asm}
  The category $\cat{Asm}$ has assemblies as objects and morphisms are
  set theoretic maps that are tracked some realizer.
\end{defn}
\begin{thm}\label{thm:systemf:asmregular}
  The $\cat{Asm}$ forms a regular category.
\end{thm}
\begin{proof}
  For a full proof see~\citet{VanOosten:08}. I will present the case
  for exponentials as it's a good illustration of the structure of
  $\cat{Asm}$.
\end{proof}
Next we shall introduce a corresponding notion of \emph{families} for
assemblies. Such a notion also is intended to capture the
computability aspects present in assemblies making them
\emph{uniform}. We shall specialize this to partial equivalence
families, PERs, on natural numbers. It is not hard to show that these
correspond to assemblies for which $x \Vdash a, b$ implies that $a =
b$.
\begin{defn}\label{defn:systemf:ufam}
  A uniform family over an assembly, $A$, is an indexed family of
  partial equivalence relations on $\mathbb{N}$ denoted
  $(R_a)_{a \in A}$. A morphism from $(A, (R_a)_{a \in A})$ to
  $(B, (S_b)_{b \in B})$ is a pair of morphisms $(u, f)$ so that
  $u \in \hom_{\cat{Asm}}(A, B)$ and a morphism $f$ so
  $f_a : R_a \to S_{f(a)}$. Moreover, $f$ must be uniformly
  tracked. This means that there is an $e$ so that $n \Vdash a$
  implies $e \cdot n \Vdash f_a$.
\end{defn}
\begin{thm}
  $\pi_1 : \cat{UFam} \to \cat{Asm}$ is a polymorphic fibration.
\end{thm}
\begin{proof}
  todo
\end{proof}

As a final note for this section, consider the subobject fibration of
a topos. It is well known that there is no naive set theoretic model
of System F~\cite{Reynolds:84}. However, it would seem that every
topos gives rise to a model of System F. So, where in is the
contradiction? The answer is the critical assumption
of~\citet{Reynolds:84} that the model isn't degenerate in that
$\denote{\lam{x}{\tau_1}{\lam{y}{\tau_1}{x}}} \neq
\denote{\lam{x}{\tau_1}{\lam{y}{\tau_1}{y}}}$. However, this will simply
not be the case in any subobject fibration. TODO: expand. like a lot.
