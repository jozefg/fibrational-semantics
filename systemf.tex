\section{Semantics for System F}

Let us start with a simple definition of System F. To be more explicit
than usual, let us start by noting the three core judgments of the
system.
\[
  \ctxJ{\Gamma} \qquad \typeJ{\Gamma}{\tau} \qquad \hastypeJ{\Gamma}{e}{\tau}
\]
The first two judgments are defined by mutual induction
\begin{mathpar}
  \inferrule{ }{\ctxJ{\cdot}} \and
  \inferrule{
    \ctxJ{\Gamma}\\
    \typeJ{\Gamma}{\tau}
  }{\ctxJ{\Gamma, x : \tau}}\and
  \inferrule{
    \typeJ{\Gamma, \alpha}{\tau}
  }{\typeJ{\Gamma}{\all{\alpha}{\tau}}}\and
  \inferrule{
    \typeJ{\Gamma}{\tau_1}\\
    \typeJ{\Gamma}{\tau_2}
  }{\typeJ{\Gamma}{\tau_1 \to \tau_2}}\and
  \inferrule{
    \alpha \in \Gamma
  }{\typeJ{\Gamma}{\alpha}}\and
\end{mathpar}

Next we have the usual typing rules for terms. One will observe that I
deviate from some presentations of System F by using only one context
to represent types \emph{and} terms rather than two distinct
contexts. This is a nod towards the semantics where such things will
be fundamentally mixed together anyways.

\begin{mathpar}
  \inferrule{
    x : \tau \in \Gamma
  }{\hastypeJ{\Gamma}{x}{\tau}}\and
  \inferrule{
    \hastypeJ{\Gamma, x : \tau_1}{e}{\tau_2}
  }{\hastypeJ{\Gamma}{\lam{x}{\tau}{e}}{\tau_1 \to \tau_2}}\and
  \inferrule{
    \hastypeJ{\Gamma}{e_1}{\tau_1 \to \tau_2}\\
    \hastypeJ{\Gamma}{e_2}{\tau_1}
  }{\hastypeJ{\Gamma}{\ap{e_1}{e_2}}{\tau_2}}\and
  \inferrule{
    \hastypeJ{\Gamma, \alpha}{e}{\tau}
  }{\hastypeJ{\Gamma}{\Lam{x}{e}}{\all{\alpha}{\tau}}}\and
  \inferrule{
    \hastypeJ{\Gamma}{e}{\all{\alpha}{\tau_1}}\\
    \typeJ{\Gamma}{\tau_2}
  }{\hastypeJ{\Gamma}{\Ap{e}{\tau_2}}{[\tau_2/\alpha]\tau_1}}\and
\end{mathpar}

Having fleshed out the objects of our language we now turn to the
semantics of our language. They will proceed in three parts. First we
isolate the specific variety of fibration that we will be
studying. Second we will denote the terms and contexts into objects in
base category. Finally, we will denote terms into the fibers above
contexts.
